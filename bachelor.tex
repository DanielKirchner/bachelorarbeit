\documentclass[a4paper,12pt,onesided]{report}
\input{macros.tex}
\begin{document}

% Titelseite
\begin{titlepage}
	\centering
	\includegraphics[width=14.9cm]{img/logo}\\

	\fontsize{18}{20}\selectfont
	Hochschule für angewandte Wissenschaften Coburg\\[.1cm]
	Fakultät Elektrotechnik und Informatik\\[1.2cm]
	Studiengang: Informatik\\
	Bachelorarbeit\\[1.2cm]
	\fontsize{21}{23}\selectfont
	\textbf{Entwicklung einer hardwarebeschleunigten Berechnung der 
	Mandelbrotmenge auf einem FPGA}\\[1cm]
	\fontsize{18}{20}\selectfont
	von\\[1.2cm]
	Daniel Kirchner\\
	Matrikelnummer: 02219415\\[1.2cm]
	Abgabe der Arbeit: 15.07.2019\\[1.2cm]

	Betreut durch: Prof. Oliver Engel, Hochschule Coburg
\end{titlepage}

\begin{abstract}
	Dieses Dokument befasst sich mit der Entwicklung einer in VHDL beschriebenen Hardware zur Darstellung
	der Mandelbrotmenge.
	Alle Berechnungen zur Visualisierung dieses Fraktals finden in Hardware statt, was die parallelen Fähigkeiten
	eines FPGAs ausnutzen lässt.
	Das Projekt wurde für das Zybo Zynq-7000 Trainer Board entwickelt.
	%TODO mehr
\end{abstract}

% Inhaltsverzeichnis
{
  \setlength{\cftbeforechapskip}{-.5ex}
  \tableofcontents
  \addcontentsline{toc}{chapter}{Inhaltsverzeichnis}
}

% Abbildungsverzeichnis
\newpage
\listoffigures
\addcontentsline{toc}{chapter}{Abbildungsverzeichnis}

% Codeverzeichnis
\newpage
\lstlistoflistings
\addcontentsline{toc}{chapter}{Codebeispielverzeichnis}

% Abkuerzungsverzeichnis
\newpage
\section*{Abkürzungsverzeichnis}
\addcontentsline{toc}{chapter}{Abkürzungsverzeichnis}
\begin{tabular}{ll}
  JAX-RS&Java API for RESTful Web Services\\
  % TODO: updaten
\end{tabular}

% Beginn Arbeit
\newpage
\chapter{Einleitung}
%TODO vllt aufbau erklären

\chapter{Grundlagen}
\section{Fraktale}
Der Begriff \textit{Fraktal} wurde maßgeblich von Benoît Mandelbrot geprägt, welcher diesen aus dem lateinischen Adjektiv
\textit{fractus} (irregulär, in Stücke zerbrochen) ableitete \cite{mandelbrot2013fraktale}. Im folgenden Kapitel sollen sowohl
die Definition eines Fraktals aufgezeigt werden, als auch einige Beispiele für derartige Strukturen aufgezeigt werden.
\subsection{Hausdorff-Dimension}
Zur mathematischen Definition eines Fraktals muss die sog. Hausdorff-Dimension erläutert werden. 
Dieses Maß ordnet einem metrischen Raum \footnote{Ein metrischer Raum ist ein Raum, in dem zwei Punkten
stets ein positiver Abstand zugeordnet werden kann.} einen Dimensionswert zu.
Um dieses Maß zu ermitteln %TODO


\section{Die Mandelbrot-Menge}
Die Mandelbrot-Menge ist eine nach Benoît Mandelbrot benannte Menge komplexer Zahlen,
welche durch Iteration 

% Literaturverzeichnis
\bibliography{bib}{}
\bibliographystyle{plain}

\end{document}