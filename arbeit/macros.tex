%Eigene Änderungen:
\usepackage[ngerman]{babel}
\usepackage{cite}
\usepackage[addtotoc]{abstract} %mit hinzufügen ins inhaltsverzeichnis
\usepackage[utf8]{inputenc}
\usepackage[T1]{fontenc}
\usepackage{graphicx}
\usepackage{hyperref}
\hbadness=99999 %abstellen nerviger warnung
\usepackage{epigraph}
\hyphenchar\font=\string"7F
\hyphenation{Man-del-brot-meng-e}
\graphicspath{ {./img/} }
%===========================================================
%== Definitionen fuer die Formatierung =====================
%===========================================================

% Use german names like "Literaturverzeichnis" instead of "Bibliography"


% Links inside of the document
%\usepackage{hyperref}

% Allows to use graphics
\usepackage{graphicx}

% Use times new roman and courier as fonts
\usepackage{times}
\usepackage{courier}

% Allow forcing positioning of floating figures
\usepackage{float}

% Allow special tye of configurable tables
\usepackage{tabularx}
\usepackage{multirow}


% Allows to change letter spcaing
\usepackage{microtype}

% Special table types
\usepackage{longtable}
\usepackage{colortbl}
\definecolor{gray}{gray}{0.85}


\usepackage[titles]{tocloft}


\usepackage{url}

\usepackage{listings}
\renewcommand*{\lstlistlistingname}{Codebeispielverzeichnis}
\renewcommand{\lstlistingname}{Code}
\lstset{
frame=single,
basicstyle=\footnotesize\bfseries\ttfamily,
breaklines=true,
numbers=left
}
\lstset{literate=%
    {Ö}{{\"O}}1
    {Ä}{{\"A}}1
    {Ü}{{\"U}}1
    {ß}{{\ss}}1
    {ü}{{\"u}}1
    {ä}{{\"a}}1
    {ö}{{\"o}}1
    {~}{{\textasciitilde}}1
}

\usepackage{titlesec}
\titlespacing*{\chapter}{0pt}{-35pt}{0pt}

\lstset{numberbychapter=false}
\usepackage{chngcntr}
\counterwithout{figure}{chapter}
\counterwithout{table}{chapter}
\renewcommand{\tablename}{Tab.}

% Glossary
% More info here https://en.wikibooks.org/wiki/LaTeX/Glossary
\usepackage[toc]{glossaries}
\renewcommand*{\glossaryentrynumbers}[1]{}
\makeglossaries

% Some formatting stuff for "Praxisbericht"
\linespread{1.4}
\makeatletter
 \renewcommand*\l@section{\@dottedtocline{1}{0em}{2.3em}}
 \renewcommand*\l@subsection{\@dottedtocline{1}{1em}{2.3em}}
\makeatother
\usepackage[a4paper, left=2.5cm, right=2.5cm, top=2.5cm, bottom=2.5cm]{geometry}

\usepackage{fancyhdr}
\pagestyle{fancy}
\fancyfoot{}
\fancyfoot[R]{\thepage}
\fancyhead{}
\fancyhead[L]{\nouppercase{\leftmark}}
\fancypagestyle{plain}{\pagestyle{fancy}}



% for formatting titles of chapters
\usepackage{titlesec}

\titleformat{\chapter} % command
{\bfseries} % format
{\fontsize{16}{18}\selectfont \thechapter\ } % label
{0ex} % sep
{\fontsize{16}{18}\selectfont} % before-code
[] % after-code
\titleformat{\section} % command
{\bfseries} % format
{\fontsize{14}{16}\selectfont \thesection\ } % label
{0ex} % sep
{\fontsize{14}{16}\selectfont} % before-code
[] % after-code
\titleformat{\subsection} % command
{\bfseries} % format
{\fontsize{12}{14}\selectfont \thesubsection\ } % label
{0ex} % sep
{\fontsize{12}{14}\selectfont} % before-code
[] % after-code
 
\usepackage{caption}
\captionsetup{
  justification=centering,
  singlelinecheck=false
}

% Paragraph styles
\setlength{\parindent}{0cm}
\setlength{\parskip}{6pt}

% A todo macro for marking ToDos
\usepackage{color}
\newcommand{\todo}[1]{\textcolor{white}{\colorbox{red}{ To do %
       :}}\textcolor{red}{\ \ #1
   }\textcolor{red}{\colorbox{red}{III}}\ }



