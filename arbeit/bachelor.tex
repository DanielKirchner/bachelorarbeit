\documentclass[a4paper,12pt,onesided]{report}
\input{macros.tex}
\begin{document}

% Titelseite
\begin{titlepage}
	\centering
	\includegraphics[width=14.9cm]{img/logo}\\

	\fontsize{18}{20}\selectfont
	Hochschule für angewandte Wissenschaften Coburg\\[.1cm]
	Fakultät Elektrotechnik und Informatik\\[1.2cm]
	Studiengang: Informatik\\
	Bachelorarbeit\\[1.2cm]
	\fontsize{21}{23}\selectfont
	\textbf{Entwicklung einer hardwarebeschleunigten Berechnung der 
	Mandelbrotmenge auf einem FPGA}\\[1cm]
	\fontsize{18}{20}\selectfont
	von\\[1.2cm]
	Daniel Kirchner\\
	Matrikelnummer: 02219415\\[1.2cm]
	Abgabe der Arbeit: 15.07.2019\\[1.2cm]

	Betreut durch: Prof. Oliver Engel, Hochschule Coburg
\end{titlepage}

\begin{abstract}
	Im Rahmen dieser Bachelorarbeit wurde eine hardwarebeschleunigte Visualisierung der Mandelbrotmenge auf einem FPGA
	realisiert.\\
	Hierfür werden diverse mathematische und designtechnische Performanceoptimierungen vorgestellt, welche dann in ein
	paralleles FPGA-Design implementiert wurden. 
	Weiterhin sollen einige Eigenschaften und Besonderheiten der Mandelbrotmenge und von Fraktalen im Allgemeinen aufgezeigt
	werden.\\
	Das Projekt wurde für das Zybo Zynq-7000 Trainer Board entwickelt, welches über einen VGA-Output die Repräsentation
	des Fraktals in Form eines 800x600@60Hz Videosignals ausgibt. Zur optimalen Ausnutzung der auf diesem Board gegebenen
	Ressourcen (DSPs, BRAM) wurde die \textit{Vivado Design Suite} mit dem integrierten IP-Katalog verwendet.\\
\end{abstract}

% Inhaltsverzeichnis
{
  \setlength{\cftbeforechapskip}{-.5ex}
  \tableofcontents
  \addcontentsline{toc}{chapter}{Inhaltsverzeichnis}
}

% Abbildungsverzeichnis
\newpage
\listoffigures
\addcontentsline{toc}{chapter}{Abbildungsverzeichnis}

% Codeverzeichnis
\newpage
\lstlistoflistings
\addcontentsline{toc}{chapter}{Codebeispielverzeichnis}

% Abkuerzungsverzeichnis
\newpage
\section*{Abkürzungsverzeichnis}
\addcontentsline{toc}{chapter}{Abkürzungsverzeichnis}
\begin{tabular}{ll}
  JAX-RS&Java API for RESTful Web Services\\
  % TODO: updaten
\end{tabular}

% Beginn Arbeit
\newpage
\chapter{Einleitung}
%TODO vllt aufbau erklären

\chapter{Grundlagen}
\section{Fraktale}
Der Begriff \textit{Fraktal} wurde maßgeblich von Benoît Mandelbrot geprägt, welcher diesen aus dem lateinischen Adjektiv
\textit{fractus} (irregulär, in Stücke zerbrochen) ableitete \cite{mandelbrot2013fraktale}. 



% Literaturverzeichnis
\bibliography{bib}{}
\bibliographystyle{plain}

\end{document}